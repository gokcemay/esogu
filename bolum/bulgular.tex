\lipsum[1-2]
\section{Örnekler}
Eğer metin içinde tabloya referans vermek isterseniz Tablo \ref{tab:den} yazarsınız. 
% ----TABLO Örneği--------
\begin{table}
\centering
\caption{Deneme Tablosu.}\label{tab:den}
\begin{tabular}{|l|l|l|}
\hline
sıra   & sayı   & toplam \\ \hline
1      & 2      & 3      \\ \hline
Kelime & deneme & son    \\ \hline
\end{tabular}
\end{table}


\subsection{Matematik}
Eğer metin içinde \(\lim_{x \to \infty} \exp(-x) = 0\) ya da ortalayabilir
\begin{displaymath}
\cos (2\theta) = \cos^2 \theta - \sin^2 \theta
\end{displaymath}
isterseniz de numaralı denklem yazabilirsiniz.

\begin{equation}
\frac{\mathrm d}{\mathrm d x} \left( k g(x) \right)
\end{equation}
\subsubsection{Kimya}
\ce{B4C} yazabilirsiniz. Ya da

\ce{CO2 + C -> 2CO}

Daha fazlası için mhchem paketine bakınız.
 
\section{Analiz}
Eğer metin içinde şekile referans vermek isterseniz Şekil\ref{fig:PtaTorc} yazarsınız. 

\bxfigure[h]
{PTA Torç \label{fig:PtaTorc}}
{\includegraphics[width=\textwidth]{gorseller/ptaTorc}}
\lipsum[1-2]
Kaynakça böyle verilebilir \parencite{celik_microstructure_2013} ya da iki yazarlı ise böyle verilebilir \parencite{gatto_plasma_2004} veya ikiden fazla ise böyle verilebilir.
\parencite{celik_effects_2011}

Kaynakça listesi için daha çok referans verilmek istenirse \parencite{yazdi_microstructure_2015, keehan_influence_2006, guo_microstructure_2014}, \parencite{kim_variation_2013}, bir başkası \parencite{xibao_metallurgical_2005},  ya da başkası \parencite{jin_effect_1997} kullanılabilir.